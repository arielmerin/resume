
%% If you want to use \orcid or the
%% academicons icons, add "academicons"
%% to the \documentclass options. 
%% Then compile with XeLaTeX or LuaLaTeX.
% \documentclass[10pt,a4paper,academicons]{altacv}
\documentclass[10pt,letterpaper]{altacv}

%% AltaCV uses the fontawesome and academicon fonts
%% and packages. 
%% See texdoc.net/pkg/fontawecome and http://texdoc.net/pkg/academicons for full list of symbols.
%% When using the "academicons" option,
%% Compile with LuaLaTeX for best results. If you
%% want to use XeLaTeX, you may need to install
%% Academicons.ttf in your operating system's font %% folder.


% Change the page layout if you need to
\geometry{left=1cm,right=9cm,marginparwidth=6.8cm,marginparsep=1.2cm,top=1cm,bottom=1cm}

% Change the font if you want to.

% If using pdflatex:
\usepackage[utf8]{inputenc}
\usepackage[spanish]{babel}
\usepackage[default]{lato}

% If using xelatex or lualatex:
% \setmainfont{Lato}

% Change the colours if you want to
\definecolor{VividPurple}{HTML}{2E64FE}
\definecolor{SlateGrey}{HTML}{2E2E2E}
\definecolor{LightGrey}{HTML}{666666}
\colorlet{heading}{VividPurple}
\colorlet{accent}{VividPurple}
\colorlet{emphasis}{SlateGrey}
\colorlet{body}{LightGrey}

% Change the bullets for itemize and rating marker
% for \cvskill if you want to
\renewcommand{\itemmarker}{{\small\textbullet}}
\renewcommand{\ratingmarker}{\faCircle}

%% sample.bib contains your publications
\addbibresource{sample.bib}

\begin{document}
\name{Kevin Ariel Merino Peña}
  \tagline{Estudiante de Ciencias de la computación}
% Cropped to square from https://en.wikipedia.org/wiki/Marissa_Mayer#/media/File:Marissa_Mayer_May_2014_(cropped).jpg, CC-BY 2.0
\photo{2.5cm}{ari}
\personalinfo{%
  % Not all of these are required!
  % You can add your own with \printinfo{symbol}{detail}
  \email{arielmerino@ciencias.unam.mx}
  \phone{+52 55 2106 4333}
  \mailaddress{Tecalli 35 , Chimalhuacán Edoméx, 56366}
  \homepage{arielmerin.github.io/}
  \twitter{@Arielmerineau}
  \linkedin{linkedin.com/in/arielmerino/}
%   \github{} % I'm just making this up though.
%   \orcid{orcid.org/0000-0000-0000-0000} % Obviously making this up too. If you want to use this field (and also other academicons symbols), add "academicons" option to \documentclass{altacv}
}

%% Make the header extend all the way to the right, if you want. Extend the right margin by 8cm (=6.8cm marginparwidth + 1.2cm marginparsep)
\begin{adjustwidth}{}{-8cm}
\makecvheader
\end{adjustwidth}

%% Provide the file name containing the sidebar contents as an optional parameter to \cvsection.
%% You can always just use \marginpar{...} if you do
%% not need to align the top of the contents to any
%% \cvsection title in the "main" bar.
\cvsection[page1sidebar]{Educación}

\cvevent{Licenciatura en Ciencias de la Computación}{Facultad de Ciencias, UNAM}{Agosto 2019 -- en curso}{Ciudad Universitaria, Coyoacán}

\divider

\cvevent{Bachillerato General}{Escuela Nacional Preparatoria No 2 \ " \ Erasmo Castellanos Quinto"}{Generación 2016}{ Av. Río Churubusco 1418}
\begin{itemize}
\item 1$^{er} $ Lugar Premio Oratoria (UNISEF- ERASMUN)
\item 2$^{do} $ Lugar Interpreparatoriano en Informática
\item 3$^{er} $ Lugar Interpreparatoriano en Educación para la Salud
\item 3$^{er}$ Lugar Olimpliada etimológica (Griego)
\end{itemize}


\cvsection{Experiencia}

\cvevent{Monitorista y logística}{Amazon DMX1, DMX3}{Junio 2019 -- Agosto 2019}{Azcapotzalco, CDMX}

\divider

\cvevent{Ventas y atención al cliente}{Grupo Thor}{Junio 2018 -- Agosto 2018}{Chimalhuacán, Edo. Méx.}

\cvsection{Filosofía de vida}
\begin{quote}
	``La vida no se trata de encontrarse a uno mismo, sino de crearse a sí mismo.''
\end{quote}


\cvsection{Un día en mi vida}

% Adapted from @Jake's answer from http://tex.stackexchange.com/a/82729/226
% \wheelchart{outer radius}{inner radius}{
% comma-separated list of value/text width/color/detail}
\wheelchart{1.5cm}{0.5cm}{%
  10/13em/accent!30/Dorimir \& descansar, 
  25/9em/accent!60/Aprendiendo \& tomando cursos online,
  5/12em/accent!10/Programar y resolver retos, 
  20/12em/accent!40/Cocinar \& ejercitarme,
  5/8em/accent!20/Sesiones de estudio en grupo (Nunca es demasiado),
  30/9em/accent/Realizando proyectos personales en computación,
  5/8em/accent!20/Redes sociales y entretenimiento
}

\cvsection{Fortalezas}

\cvtag{Autodidacta}
\cvtag{Trabaja duro} 
\cvtag{Comunicación asertiva}
\cvtag{Proactivo} 

\divider\smallskip

\cvtag{Trabajo en equipo}
\cvtag{Análisis estadístico \& ensayo de hipótesis}
\cvtag{Graph Theory}
\cvtag{Algoritmos}

\clearpage


\end{document}
